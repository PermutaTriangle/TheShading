\documentclass[12pt,reqno]{amsart}
\usepackage{amsfonts, amssymb, latexsym, graphics, color, longtable}

\usepackage{stmaryrd}		% various mathematical symbols

\usepackage{algorithmic}
\usepackage{algorithm}

\setlength{\oddsidemargin}{0in}
\setlength{\evensidemargin}{0in}
\setlength{\marginparwidth}{0in}
\setlength{\marginparsep}{0in}
\setlength{\marginparpush}{0in}
\setlength{\topmargin}{0in}
\setlength{\headheight}{0in}
\setlength{\headsep}{.3in}
\setlength{\footskip}{.3in}
\setlength{\textheight}{8.7in}
\setlength{\textwidth}{6.3in}
\setlength{\parskip}{4pt}

\theoremstyle{plain}
\newtheorem{theorem}{Theorem}[section]
\newtheorem{proposition}[theorem]{Proposition}
\newtheorem{observation}[theorem]{Observation}
\newtheorem{corollary}[theorem]{Corollary}
\newtheorem{lemma}[theorem]{Lemma}

\theoremstyle{definition}
\newtheorem{definition}[theorem]{Definition}
\newtheorem{conjecture}[theorem]{Conjecture}
\newtheorem{example}[theorem]{Example}
\newtheorem{remark}[theorem]{Remark}

\numberwithin{equation}{section}

\raggedbottom % Makes the bottom margin more flexible (helpful for pictures)

% For tikz pictures
\usepackage{tikz}
\usetikzlibrary{matrix,arrows}
\usetikzlibrary{positioning}
\usetikzlibrary{fit}
\usetikzlibrary{patterns}
\usetikzlibrary{arrows}

%%%%%%%%%%% MACROS FOR DRAWING MESH PATTERNS %%%%%%%%%%%

\newcommand{\pattern}[4]{										% mesh pattern
  \raisebox{0.6ex}{
  \begin{tikzpicture}[scale=0.35, baseline=(current bounding box.center), #1]
  \useasboundingbox (0.0,-0.1) rectangle (#2+1.4,#2+1.1);
    \foreach \x/\y in {#4}
      \fill[pattern color = black!65, pattern=north east lines] (\x,\y) rectangle +(1,1);
    \draw (0.01,0.01) grid (#2+0.99,#2+0.99);
    \foreach \x/\y in {#3}
      \filldraw (\x,\y) circle (6pt);
  \end{tikzpicture}}
}
%%%%%%%%%%%%%%%%%%%%%%%%%%%%%%%%%%%%%%%%%%%%%%%%%%%%%%%%%%%%%%%%%%

%\usepackage[pdftex]{hyperref}


\begin{document}

\title{Unexplained coincidences for $PERMUTATION$}

\author[Tenner]{Bridget Tenner}
\author[Claesson]{Anders Claesson}
\author[Ulfarsson]{Henning Ulfarsson}

\address[Tenner]{School of Computer Science, Reykjav\'ik University, Menntavegi 1, 101 Reykjav\'ik, \mbox{Iceland}}
\address[Claesson]{Department of Computer and Information Sciences, University of Strathclyde, Glasgow G1 1XH, UK}
\address[Ulfarsson]{School of Computer Science, Reykjav\'ik University, Menntavegi 1, 101 Reykjav\'ik, Iceland}

\email{anders.claesson@cis.strath.ac.uk, henningu@ru.is}

%\thanks{Ulfarsson is supported by grant no.\ 090038013 from the Icelandic Research Fund.}

%\date{\today}
\begin{abstract}
The following coincidences are found by first applying the Shading Algorithm to
get all coincidence classes on the pattern $PERMUTATION$. Then for each class we pick (one of) the least shaded pattern(s)
and compute the avoiding permutations of lengths $0, \dots, LENGTH$. If two (or more) classes turn out to have the same avoiding permutations they
are listed together below.
\end{abstract}

\keywords{Permutation Patterns}

% No need to include the dates
%\received{(Some date $\leq$ 1$^{\mbox{\footnotesize{st}}}$ December 2009)}
%\revised{\today}
%\accepted{tomorrow}

\maketitle

\section{Parameters used by the Shading Algorithm}
PARAMS

\section{Coincidences}
COINCIDENCES

\end{document}
